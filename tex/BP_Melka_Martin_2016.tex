% options:
% thesis=B bachelor's thesis
% thesis=M master's thesis
% czech thesis in Czech language
% slovak thesis in Slovak language
% english thesis in English language
% hidelinks remove colour boxes around hyperlinks

\documentclass[thesis=B,czech]{FITthesis}[2012/06/26]

\usepackage[utf8]{inputenc} % LaTeX source encoded as UTF-8

\usepackage{graphicx} %graphics files inclusion
% \usepackage{amsmath} %advanced maths
% \usepackage{amssymb} %additional math symbols
\usepackage[figuresleft]{rotating}
\usepackage[flushleft]{threeparttable} % for footnote in tabular

\usepackage{dirtree} %directory tree visualisation

% % list of acronyms
% \usepackage[acronym,nonumberlist,toc,numberedsection=autolabel]{glossaries}
% \iflanguage{czech}{\renewcommand*{\acronymname}{Seznam pou{\v z}it{\' y}ch zkratek}}{}
% \makeglossaries

\newcommand{\tg}{\mathop{\mathrm{tg}}} %cesky tangens
\newcommand{\cotg}{\mathop{\mathrm{cotg}}} %cesky cotangens

% Macros to use as research features criteria. 
% The text is used repeatedly, so in case of its change I want the change
% to happen everywhere.
\newcommand{\crita}{Nevyžaduje vlastní infrastrukturu, rychlé zprovoznení}
\newcommand{\critb}{Jednoduché používání}
\newcommand{\critc}{Absence nepotrebné funkcionality}
\newcommand{\critd}{Použití zdarma}
\newcommand{\crite}{Funkce}
\newcommand{\critf}{REST API}
\newcommand{\critg}{Open-source}

\department{Katedra Softwarového Inženýrství}
\title{Systém správy úkolů pro jednotlivce a malé týmy}
\authorGN{Martin} %(křestní) jméno (jména) autora
\authorFN{Melka} %příjmení autora
\authorWithDegrees{Martin Melka} %jméno autora včetně současných akademických titulů
\supervisor{Ing. Josef Pavlíček, Ph.D.}
\acknowledgements{Doplňte, máte-li komu a za co děkovat. V~opačném případě úplně odstraňte tento příkaz.}

\abstractCS{Tato bakalářská práce se zabývá srovnáním existujících aplikací a tvorbou nové aplikace pro správu úkolů. Uživateli této aplikace budou jednotlivci a menší pracovní skupiny, a přidělovat a sledovat průběh práce na společných úkolech. Aplikace umožní lidem sdružovat se do skupin, spolupracovat na sdílených úkolech a zaznamenávat odvedenou práci. Součástí této práce je definice požadavků na aplikaci, srovnání navrhovaného řešení s existujícími aplikacemi, dále návrh, implementace, testování a nasazení aplikace. Výsledkem práce bude aplikační backend, vystavující funkcionalitu skrze REST rozhraní.}

\abstractEN{The aim of this thesis is to compare available software applications for task management and to subsequently create an original one. The users of this application will be individuals and small-scale workgroups, who need to assign responsibilities for and track the progress of shared tasks. The application will allow users to form groups, work together on shared tasks and report the work done on them. This thesis consists of a definition of application requirements, comparison of current task management solutions and design, implementation, testing and deployment of the proposed application. The result of this thesis will be an application backend, which exposes its functionality through a REST interface.}
\placeForDeclarationOfAuthenticity{V~Praze}
\declarationOfAuthenticityOption{4} %volba Prohlášení (číslo 1-6)
\keywordsCS{Správa úkolů, produktivita, organizace týmů, backend, REST, Java Spring Framework}
\keywordsEN{Task management, productivity, team organization, backend, REST, Java Spring Framework}

\begin{document}

% \newacronym{CVUT}{{\v C}VUT}{{\v C}esk{\' e} vysok{\' e} u{\v c}en{\' i} technick{\' e} v Praze}
% \newacronym{FIT}{FIT}{Fakulta informa{\v c}n{\' i}ch technologi{\' i}}

\begin{introduction}
	Výpočetní technika umožnila rozvoj rychlejší a efektivnější komunikace, práce a vůbec způsobu života. Je běžné mít svůj kalendář on-line a sdílet ho s ostatními, případně používat některý nástroj s funkcí úkolníčku. Zejména ve velkých firmách, kde existuje silná potřeba koordinovat úsilí mnoha lidí, tak vznikla poptávka po nástrojích pro správu úkolů, které by jim umožnily efektivnější rozdělování zodpovědností a práce. Řešení, které na tento popud vznikly, slouží právě potřebám velkých firem. Potřeby menších skupin a jednotlivců jsou ale jiné. Pro ně jsou tyto nástroje příliš komplexní, těžkopádné a nedostatečně intuitivní.

Tato práce se zabývá přehledem existujících řešení pro správu úkolů malých i velkých skupin a tvorbou backendu nové aplikace. Aplikace bude zaměřena na potřeby menších týmů a jednotlivců a bude umožňovat uživatelům vytvářet úkoly, sdílet je s ostatními uživateli, sdružovat se do skupin a zaznamenávat průběh práce na úkolech. 

V první části porovnám současná řešení pro správu úkolů z pohledu malých týmů a jednotlivců. 

Ve druhé části se zabývám analýzou problému. Na základě kladů a záporů jednotlivých řešení zmíněných v části první vytvořím požadavky na aplikaci ve formě uživatelských příběhů a tyto požadavky podrobím analýze.

Ve třetí části vypracuji návrh řešení aplikace. Zde vybírám technologii pro implementaci a popisuji architekturu aplikace, návrhový model tříd a databázový model.

Ve čtvrté části popisuji implementaci aplikace.

V páté části se věnuji testování aplikace za účelem zjištění její správné funkčnosti. Popíšu zde použité technologie a způsob testování.

V poslední, šesté části, vysvětlím, jak výslednou aplikaci nasadit a spustit.
\end{introduction}

\chapter{Současný stav}
	\label{chapter:current-state}

	Způsobů, jak řešit správu úkolů, existuje spoustu a liší se podle toho, kdo je má využívat. V této části práce představím několik zástupců pro každou ze tří kategorií. Těmi jsou:
	\begin{enumerate}
	  \item Řešení pro větší firmy s množstvím pracovníků
	  \item Řešení pro střední a menší týmy, jednotky až desítky pracovníků
	  \item Řešení pro jednotlivce
	\end{enumerate}

	Zástupce vybírám podle osobních zkušenosti a podle výsledků získaných z vyhledávače \texttt{Google}, na základě jejich pořadí a popularity mezi uživateli. U uvedených zástupců zvážím klady a zápory jak pro jejich cílovou skupinu, tak pro cílovou skupinu této práce, tj. menší týmy a jednotlivce.

	Pracuji s tím, že pro cílovou skupinu této práce jsou důležitá následující kritéria:
	\begin{enumerate}
		\item \textbf{\crita} -- Uživatelé nechtějí spravovat vlastní hardware, kde by jim aplikace běžela. Začít používat aplikaci má být otázka maximálně minut.
		\item \textbf{\critb} -- Relativně subjektivní kritérium; uživatel by se neměl ztratit ve funkcích aplikace a měl by být schopen rychle pochopit, jak s aplikací pracovat.
		\item \textbf{\critc} -- 	Aplikace by měla obsahovat jen základní funkce, které uživatel využije. Větší množství funkcí, které uživatele nezajímají, ztěžují orientaci v aplikaci, což souvisí s předchozím bodem.
		\item \textbf{\critd} -- Aplikace by měla být použitelná zdarma. V případě, že se jedná o \textit{freemium} model, měla by její neplacená část stačit k běžnému používání a neomezovat výrazně uživatele.
		\item \textbf{\crite} -- Aplikace má umožňovat spravovat úkoly jak pro jednotlivce, tak týmy. Mezi požadavky na funkce tedy patří možnost vytvářet osobní i sdílené úkoly, u nich uvádět datum \textit{deadlinu}, pracnost, prioritu; dále sdružovat se do skupin, v rámci skupin přiřazovat odpovědnosti za úkoly a zaznamenávat odpracovanou práci jednotlivých uživatelů. Aplikace by také měla uživateli doporučit, kterému úkolu se věnovat, v závislosti na jeho prioritě, pracnosti a času zbývajícímu do deadlinu. Úkol také nemusí mít deadline a jeho priorita sama postupně časem poroste.
		\item \textbf{\critf} -- Aplikace by měla nabízet rozhraní REST API pro možnost vlastní integrace na její funkce.
		\item \textbf{\critg} -- Aplikace by měla mít veřejně dostupné zdrojové kódy.
	\end{enumerate}
	Uvedený přehled není vyčerpávající, věnuji se jen některým z těch nejznámějších řešení. 

	\section{Řešení pro firmy}
		\label{sec:solutions-companies}
		Řešení této kategorie se zaměřují na větší počet uživatelů a mimo základní správy úkolů nabízí často mnoho dalších funkcí pro řízení projektů a integraci s dalšími systémy. Používají se zejména v oblasti vývoje software, ale dají se využít i v jiných oblastech.
		
		\subsection{JIRA}
			JIRA \cite{jira} je software, který nabízí bug tracking, issue tracking a funkce pro správu projektů. Je možné ho používat jak na vlastní HW infrastruktuře, tak on-line. V prvním případě je použití zdarma za určitých podmínek\footnote{Zdarma pro veřejně dostupný open-source software projekt\cite{jira-lic-opensource} a pro neziskové, nevládní, neakademické, nekomerční a sekulární instituce, které by si jinak nemohly software dovolit. \cite{jira-lic-nonprofit}}, v druhém případě je použití placené. 
			
			Nabízí širokou funkcionalitu a např. možnost upravovat podle potřeb životní cyklus úkolů. To ho činí využitelným i mimo vývoj software. Množství nabízených funkcí jde ale nad potřeby cílové skupiny této práce a technicky méně zdatné uživatele může mást. Na úkolech lze pracovat ve více lidech, ale ne najednou (\textit{assignee} může být v jednu chvíli jen jeden uživatel). Úkolům lze přiřadit deadline, ale uživatelé nemůžou dostat doporučení, na kterém úkolu by měli pracovat.
			
			JIRA nabízí REST API a je closed-source.
			
		\subsection{Bugzilla}
			Bugzilla \cite{bugzilla} je bug tracking nástroj, který je zaměřen hlavně na vývoj software. Je podobný nástroji JIRA, nicméně nenabízí takovou flexibilitu a i když by mohl být s dobře nastavenou politikou použitý pro správu úkolů u jiných než softwarových projektů, nebylo by použití intuitivní. 
			
			Bugzilla je open-source, licencovaná pod MPL, a lze ji využít zdarma i pro komerční účely. Je nutné ji ale provozovat na vlastním hardware. Existují hosting služby, které jsou ale neoficiální a placené. Bugzilla nabízí REST API.
			
		\subsection{Redmine}		
			Redmine \cite{redmine} je issue tracking nástroj, který nabízí více flexibility než Bugzilla a obsahuje i některé nástroje pro řízení projektů. Tyto nástroje mohou být přínosné pro větší projekty, které mají danou strukturu, ale nepočítám s tím, že by cílovou skupinu této práce zajímaly. 
			
			Použití je zdarma, ale je nutné nainstalovat na vlastním hardware. Stejně jako v případě Bugzilly není Redmine oficiálně použitelný on-line, soukromé hostingy jsou placené. Projekt je vyvíjen jako open-source a nabízí REST API.

	\section{Řešení pro střední a menší týmy}
		\label{sec:solutions-teams}
		Nástroje v této kategorii se snaží cílit na týmy, spíše než celé firmy, a práce s nimi není tak formální. Používat je lze on-line, není nutné vlastní instalace. Oproti řešením v bodu \ref{sec:solutions-companies} obsahují tyto méně funkcí, chybí hlavně různé manažerské nástroje a integrace s dalšími systémy.
	
		\subsection{Trello}
			Trello \cite{trello} je on-line aplikace, která vznikla v roce 2011. Způsob správy úkolů staví na konceptu \textit{kanban}\cite{kanban}. Umožňuje vytvářet nástěnky (boards), které reprezentují projekty. K nástěnkám lze přizvat další uživatele a pracovat na nich společně. Na nástěnkách se dají vytvářet seznamy (lists) a v nich karty (cards), které představují nejmenší jednotku práce - úkol. Ten má svou prioritu, deadline a zodpovědné uživatele.
			
			Funkčně bohatý nástroj, s jednoduchým ovládáním i pro netechnické uživatele. Na profilu uživatele lze zobrazit všechny přiřazené karty a ty seřadit podle nástěnky, kam patří, nebo deadlinu. U karet ale není možné uvést odhad pracnosti a aplikace tak nedokáže radit, které kartě by se měl uživatel věnovat, aby ji do deadlinu stihl.
			
			Trello je možné používat zdarma s libovolným počtem spolupracovníků. Placené varianty přináší určité výhody\cite{trello-pricing}, ale menší týmy se moho obejít bez nich. Existuje i REST API\cite{trello-api}, ale zdrojové kódy nejsou dostupné.
		
			
		\subsection{Trackie}
			Trackie \cite{trackie} je on-line aplikace, která je určena pro správu úkolů na společných projektech. Ty lze zakládat a zvát do nich uživatele, v rámci projektů pak tvořit úkoly. Úkol může být někomu přiřazen, ale vždy jen jednomu uživateli. Funkčností i vzhledem jednoduché na použití, ale některé funkce chybí. Úkolům například nelze nastavit deadline ani odhad trvání. Zobrazit je možné jen úkoly každého projektu zvlášť, chybí přehled všech úkolů uživatele. 
			
			Aplikace nabízí 30denní zkušební dobu, po její uplynutí je placená. Nenabízí REST API a její zdrojové kódy nejsou veřejné.
			
		\subsection{FogBugz}
			FogBugz \cite{fogbugz} je nástroj řízení projektů, který kromě issue trackingu nabízí i možnost agilního plánování\cite{agile-planning}, správu podpory a zpětné vazby zákazníků, vytváření dokumentu ve stylu Wiki a další. Nabídkou funkcí je nejbohatší z trojice nástrojů v této kategorii, ale i přesto se s ním pracuje jednoduše. 
			
			Uživatelé spolu mohou pracovat na úkolech (cases), které se dělí do projektů (projects). Na úkolu lze evidovat všechny potřebné informace pro využití v agilní metodice plánování, včetně \textit{story points}, dále zodpovědného uživatele, odhad pracnosti a deadline. Na úkolu je možné průběžně zaznamenávat odpracovanou práci a zobrazovat kolik času na něm zbývá. Uživatel ale nedostane informaci o tom, na kterém úkolu by měl pracovat, aby ho bylo ještě možné stihnout v závislosti na času pracnosti a času zbývajícího do deadlinu.
			
			FogBugz nabízí 7denní zkušební dobu zdarma, poté je nutné za používání platit. Je možné ho používat jak on-line, tak na vlastním hardware. Nabízí REST API, ale zdrojové kódy nejsou veřejně dostupné.

	\section{Řešení pro jednotlivce}
		\label{sec:solutions-individuals}
		Poslední kategorií jsou nástroje pro správu úkolů jednotlivců. Některé z nich mohou umožňovat sdílení úkolů, takže by se daly zařadit i do kategorie \ref{sec:solutions-teams}, nicméně svým zaměřením cílí primárně na využití jako osobní úkolníček, proto jsou zařazeny zde. Také jsem do této kategorie zařadil nástroje, které nejsou primárně určeny pro správu úkolů, ale někteří je k tomuto účelu využívají, například kalendář nebo poznámky.
		
		\subsection{Todoist}
			Todoist \cite{todoist} je on-line aplikace pro organizaci úkolů, která má uživatele motivovat k lepší produktivitě. Za splněné úkoly jsou přidělovány body (karma), jejichž historický vývoj je možné sledovat v grafu, lze nastavit denní a týdenní cíl a získávat další \uv{odměny} za jeho splnění. Úkoly lze rozdělit do projektů, nastavit jim deadline, prioritu a zda se mají opakovat. Projekty je možné sdílet s dalšími uživateli a dají se zobrazit buď po projektech, ke kterým patří, nebo všechny na jednom místě -- ve schránce (inbox). 
			
			Vytvoření nového úkolu se může provést jen zadáním (anglického) textu, aplikace sama rozpozná řídící slova a není tak nutné nastavovat vlastnosti úkolů ručně. Například heslo \textit{Sometask at 1 PM every day} vytvoří úkol, který má deadline ve 13 hodin a opakuje se každý den. U úkolů ale nelze nastavit pracnost a aplikace uživateli neporadí, kterému úkolu se má věnovat.
			
			Placená verze nabízí větší množství otevřených projektů a úkolů, hledání v úkolech, notifikace a další.\cite{todoist-compare-premium} Todoist nabízí REST API\cite{todoist-api} a jeho zdrojové kódy nejsou veřejně dostupné.
			
		\subsection{Toodledo}
			Toodledo \cite{toodledo} je on-line aplikace, která kromě úkolů umožňuje vytvářet si zvyky (habits). To jsou opakující se úkoly, jež mají uživatelům pomoci vypěstovat si a dodržovat dobré návyky. Po splnění zvyku je možné ho označit za splněný, přidat k jeho splnění číslo nebo hodnocení. Mezi další možnosti patří poznámky (notes), seznamy (lists), nebo nástiny (outlines). 
			
			Úkoly lze třídit do složek, nastavit jim deadline a prioritu. Nelze ale určovat pracnost a spolupráce s dalšími uživateli vyžaduje placenou verzi aplikace. Webové rozhraní aplikace je oproti nástroji Todoist poměrně nepřehledné. Toodledo nabízí REST API\cite{toodledo-api} a zdrojové kódy nejsou veřejně dostupné.
			
		\subsection{Google Inbox Reminders}
			Inbox \cite{ginbox} je e-mailový klient od společnosti Google, který kromě práce s e-maily umožňuje vytváření jednoduchých úkolů (reminders). Ty lze také zobrazit v kalendáři Google Calendar.\cite{gcal} Úkolům nelze nastavit deadline, ale je možné je odložit (snooze) tak, aby se zobrazily později. Aktivní úkoly se zobrazují mezi příchozími e-maily.
			
			Úkoly nelze nijak třídit, určovat pracnost ani prioritu. Jediná informace v úkolu je jeho popis. Pro základní potřeby dostačující, ale jinak funkčně chudý nástroj zatím nenabízí API\cite{ginbox-no-api} a jeho zdrojové kódy nejsou dostupné.
			
			
			
		\subsection{Poznámky, kalendář}
			Tato sekce není jeden konkrétní produkt, ale typ produktů. Uvádím je pro úplnost, jelikož je stále využívá velké množství lidí. Patří sem všechny nástroje pro psaní poznámek a vytváření úkolů v kalendáři, a to jak elektronické, tak papírové. 
			
			Některé elektronické nástroje umožňují sdílení poznámek či kalendářů, takže se dají při dobře definovaných pravidlech použít i pro týmovou práci. Jejich použití je ale složitější s rostoucím počtem úkolů a projektů, protože neumožňují žádné filtrování, řazení úkolů ani přiřazování zodpovědností. Pro nenáročného uživatele, který si chce zapisovat nejdůležitější úkoly a sám se postará o to, že na nich nezapomene včas začít, může být toto řešení dostačující.
			
			Pro papírové \uv{nástroje} platí předchozí odstavec podobné, jen možnost spolupráce je ještě více omezena. Pokud celá skupina nemá společnou místnost pro práci, pak prakticky vyloučena. Navíc vzniká problém s archivací a doslovnou ztrátou úkolů.
		
	
	\section{Splnění kritérií}
		Žádný z uvedených nástrojů nesplnil všech 7 požadavků stanovených na začátku kapitoly \ref{chapter:current-state}. Všechny nástroje je možné používat bez vlastního HW, ale v dalších požadavcích se lišily. 
		
		Nejlépe dopadly nástroje ze sekce \ref{sec:solutions-teams}, jmenovitě Trello a Todoist. Chyběly jim některé funkce oproti definovaným požadavkům, zejména možnost vytvoření úkolu bez deadlinu s postupně rostoucí prioritou a doporučení uživatelům, na jakém úkolu by měli pracovat vzhledem k prioritě, zbývající pracnosti úkolu a deadlinu. Nástroje také nejsou open-source. Splnění požadavků jednotlivými je detailně zobrazeno v tabulce \ref{table:criteria}.
		
		Backend část aplikace, jež bude výstupem této práce, bude spadat na pomezí sekcí \ref{sec:solutions-teams} a \ref{sec:solutions-individuals}. Bude obsahovat vybrané funkce pro správu práce týmů i jednotlivců, které analyzuji v kapitole \ref{chapter:analysis}.


	\begin{sidewaystable}
	\begin{threeparttable}
		\caption{Tabulka splnění kritérií rešerše}
		\label{table:criteria}
		\begin{tabular}{|c||c|c|c|c|c|c|c|c|c|c|}
			\hline
			      & JIRA & Bugzilla & Redmine & Trello & Trackie & FogBugz & Todoist & Toodledo & Inbox & Poznámky \\ \hline \hline
			Kr. 1 & + & + & + & +  & + & +  & + & +   & + & +    \\ \hline
			Kr. 2 & ~ & ~ & ~ & +  & + & +  & + & ~   & + & +    \\ \hline
			Kr. 3 & ~ & ~ & ~ & +  & + & ~  & + & +   & ~ & +    \\ \hline
			Kr. 4 & ~ & * & * & +  & ~ & ~  & + & *** & + & +    \\ \hline
			Kr. 5 & ~ & ~ & ~ & ** & ~ & ** & ~ & ~   & ~ & ~    \\ \hline
			Kr. 6 & + & + & + & +  & ~ & +  & + & +   & ~ & **** \\ \hline
			Kr. 7 & ~ & + & + & ~  & ~ & ~  & ~ & ~   & ~ & **** \\
			\hline
		\end{tabular}
		
		\begin{tablenotes}
			\item[+] Splněno
			\item[*] Zdarma na vlastním HW.
			\item[**] Velká část splněna
			\item[***] Zdarma jen bez spolupráce na úkolech
			\item[****] Záleží na konkrétním produktu
			\item Kr. 1 -- \crita
			\item Kr. 2 -- \critb
			\item Kr. 3 -- \critc
			\item Kr. 4 -- \critd
			\item Kr. 5 -- \crite
			\item Kr. 6 -- \critf
			\item Kr. 7 -- \critg
		\end{tablenotes}
	\end{threeparttable}		
	\end{sidewaystable}
	
	
\chapter{Analýza a návrh}
	\label{chapter:analysis}

\chapter{Realizace}

\begin{conclusion}
	%sem napište závěr Vaší práce
\end{conclusion}

\bibliographystyle{csn690}
\bibliography{melka-bibliography}

\appendix

\chapter{Seznam použitých zkratek}
% \printglossaries
\begin{description}
	\item[HW] Hardware
	\item[SW] Software
\end{description}


% % % % % % % % % % % % % % % % % % % % % % % % % % % % 
% % Tuto kapitolu z výsledné práce ODSTRAŇTE.
% % % % % % % % % % % % % % % % % % % % % % % % % % % % 
% 
% \chapter{Návod k~použití této šablony}
% 
% Tento dokument slouží jako základ pro napsání závěrečné práce na Fakultě informačních technologií ČVUT v~Praze.
% 
% \section{Výběr základu}
% 
% Vyberte si šablonu podle druhu práce (bakalářská, diplomová), jazyka (čeština, angličtina) a kódování (ASCII, \mbox{UTF-8}, \mbox{ISO-8859-2} neboli latin2 a nebo \mbox{Windows-1250}). 
% 
% V~české variantě naleznete šablony v~souborech pojmenovaných ve formátu práce\_kódování.tex. Typ může být:
% \begin{description}
% 	\item[BP] bakalářská práce,
% 	\item[DP] diplomová (magisterská) práce.
% \end{description}
% Kódování, ve kterém chcete psát, může být:
% \begin{description}
% 	\item[UTF-8] kódování Unicode,
% 	\item[ISO-8859-2] latin2,
% 	\item[Windows-1250] znaková sada 1250 Windows.
% \end{description}
% V~případě nejistoty ohledně kódování doporučujeme následující postup:
% \begin{enumerate}
% 	\item Otevřete šablony pro kódování UTF-8 v~editoru prostého textu, který chcete pro psaní práce použít -- pokud můžete texty s~diakritikou normálně přečíst, použijte tuto šablonu.
% 	\item V~opačném případě postupujte dále podle toho, jaký operační systém používáte:
% 	\begin{itemize}
% 		\item v~případě Windows použijte šablonu pro kódování \mbox{Windows-1250},
% 		\item jinak zkuste použít šablonu pro kódování \mbox{ISO-8859-2}.
% 	\end{itemize}
% \end{enumerate}
% 
% 
% V~anglické variantě jsou šablony pojmenované podle typu práce, možnosti jsou:
% \begin{description}
% 	\item[bachelors] bakalářská práce,
% 	\item[masters] diplomová (magisterská) práce.
% \end{description}
% 
% \section{Použití šablony}
% 
% Šablona je určena pro zpracování systémem \LaTeXe{}. Text je možné psát v~textovém editoru jako prostý text, lze však také využít specializovaný editor pro \LaTeX{}, např. Kile.
% 
% Pro získání tisknutelného výstupu z~takto vytvořeného souboru použijte příkaz \verb|pdflatex|, kterému předáte cestu k~souboru jako parametr. Vhodný editor pro \LaTeX{} toto udělá za Vás. \verb|pdfcslatex| ani \verb|cslatex| \emph{nebudou} s~těmito šablonami fungovat.
% 
% Více informací o~použití systému \LaTeX{} najdete např. v~\cite{wikilatex}.
% 
% \subsection{Typografie}
% 
% Při psaní dodržujte typografické konvence zvoleného jazyka. České \uv{uvozovky} zapisujte použitím příkazu \verb|\uv|, kterému v~parametru předáte text, jenž má být v~uvozovkách. Anglické otevírací uvozovky se v~\LaTeX{}u zadávají jako dva zpětné apostrofy, uzavírací uvozovky jako dva apostrofy. Často chybně uváděný symbol "{} (palce) nemá s~uvozovkami nic společného.
% 
% Dále je třeba zabránit zalomení řádky mezi některými slovy, v~češtině např. za jednopísmennými předložkami a spojkami (vyjma \uv{a}). To docílíte vložením pružné nezalomitelné mezery -- znakem \texttt{\textasciitilde}. V~tomto případě to není třeba dělat ručně, lze použít program \verb|vlna|.
% 
% Více o~typografii viz \cite{kobltypo}.
% 
% \subsection{Obrázky}
% 
% Pro umožnění vkládání obrázků je vhodné použít balíček \verb|graphicx|, samotné vložení se provede příkazem \verb|\includegraphics|. Takto je možné vkládat obrázky ve formátu PDF, PNG a JPEG jestliže používáte pdf\LaTeX{} nebo ve formátu EPS jestliže používáte \LaTeX{}. Doporučujeme preferovat vektorové obrázky před rastrovými (vyjma fotografií).
% 
% \subsubsection{Získání vhodného formátu}
% 
% Pro získání vektorových formátů PDF nebo EPS z~jiných lze použít některý z~vektorových grafických editorů. Pro převod rastrového obrázku na vektorový lze použít rasterizaci, kterou mnohé editory zvládají (např. Inkscape). Pro konverze lze použít též nástroje pro dávkové zpracování běžně dodávané s~\LaTeX{}em, např. \verb|epstopdf|.
% 
% \subsubsection{Plovoucí prostředí}
% 
% Příkazem \verb|\includegraphics| lze obrázky vkládat přímo, doporučujeme však použít plovoucí prostředí, konkrétně \verb|figure|. Například obrázek \ref{fig:float} byl vložen tímto způsobem. Vůbec přitom nevadí, když je obrázek umístěn jinde, než bylo původně zamýšleno -- je tomu tak hlavně kvůli dodržení typografických konvencí. Namísto vynucování konkrétní pozice obrázku doporučujeme používat odkazování z~textu (dvojice příkazů \verb|\label| a \verb|\ref|).
% 
% \begin{figure}\centering
% 	\includegraphics[width=0.5\textwidth, angle=30]{cvut-logo-bw}
% 	\caption[Příklad obrázku]{Ukázkový obrázek v~plovoucím prostředí}\label{fig:float}
% \end{figure}
% 
% \subsubsection{Verze obrázků}
% 
% % Gnuplot BW i barevně
% Může se hodit mít více verzí stejného obrázku, např. pro barevný či černobílý tisk a nebo pro prezentaci. S~pomocí některých nástrojů na generování grafiky je to snadné.
% 
% Máte-li například graf vytvořený v programu Gnuplot, můžete jeho černobílou variantu (viz obr. \ref{fig:gnuplot-bw}) vytvořit parametrem \verb|monochrome dashed| příkazu \verb|set term|. Barevnou variantu (viz obr. \ref{fig:gnuplot-col}) vhodnou na prezentace lze vytvořit parametrem \verb|colour solid|.
% 
% \begin{figure}\centering
% 	\includegraphics{gnuplot-bw}
% 	\caption{Černobílá varianta obrázku generovaného programem Gnuplot}\label{fig:gnuplot-bw}
% \end{figure}
% 
% \begin{figure}\centering
% 	\includegraphics{gnuplot-col}
% 	\caption{Barevná varianta obrázku generovaného programem Gnuplot}\label{fig:gnuplot-col}
% \end{figure}
% 
% 
% \subsection{Tabulky}
% 
% Tabulky lze zadávat různě, např. v~prostředí \verb|tabular|, avšak pro jejich vkládání platí to samé, co pro obrázky -- použijte plovoucí prostředí, v~tomto případě \verb|table|. Například tabulka \ref{tab:matematika} byla vložena tímto způsobem.
% 
% \begin{table}\centering
% 	\caption[Příklad tabulky]{Zadávání matematiky}\label{tab:matematika}
% 	\begin{tabular}{|l|l|c|c|}\hline
% 		Typ		& Prostředí		& \LaTeX{}ovská zkratka	& \TeX{}ovská zkratka	\tabularnewline \hline \hline
% 		Text		& \verb|math|		& \verb|\(...\)|	& \verb|$...$|		\tabularnewline \hline
% 		Displayed	& \verb|displaymath|	& \verb|\[...\]|	& \verb|$$...$$|	\tabularnewline \hline
% 	\end{tabular}
% \end{table}
% 
% % % % % % % % % % % % % % % % % % % % % % % % % % % % 

\chapter{Obsah přiloženého CD}

%upravte podle skutecnosti

\begin{figure}
	\dirtree{%
		.1 readme.txt\DTcomment{stručný popis obsahu CD}.
		.1 exe\DTcomment{adresář se spustitelnou formou implementace}.
		.1 src.
		.2 impl\DTcomment{zdrojové kódy implementace}.
		.2 thesis\DTcomment{zdrojová forma práce ve formátu \LaTeX{}}.
		.1 text\DTcomment{text práce}.
		.2 thesis.pdf\DTcomment{text práce ve formátu PDF}.
		.2 thesis.ps\DTcomment{text práce ve formátu PS}.
	}
\end{figure}

\end{document}
