% options:
% thesis=B bachelor's thesis
% thesis=M master's thesis
% czech thesis in Czech language
% slovak thesis in Slovak language
% english thesis in English language
% hidelinks remove colour boxes around hyperlinks

\documentclass[thesis=B,czech]{FITthesis}[2012/06/26]

\usepackage[utf8]{inputenc} % LaTeX source encoded as UTF-8

\usepackage{graphicx} %graphics files inclusion
% \usepackage{amsmath} %advanced maths
% \usepackage{amssymb} %additional math symbols

\usepackage{dirtree} %directory tree visualisation

% % list of acronyms
% \usepackage[acronym,nonumberlist,toc,numberedsection=autolabel]{glossaries}
% \iflanguage{czech}{\renewcommand*{\acronymname}{Seznam pou{\v z}it{\' y}ch zkratek}}{}
% \makeglossaries

\newcommand{\tg}{\mathop{\mathrm{tg}}} %cesky tangens
\newcommand{\cotg}{\mathop{\mathrm{cotg}}} %cesky cotangens

% % % % % % % % % % % % % % % % % % % % % % % % % % % % % % 
% ODTUD DAL VSE ZMENTE
% % % % % % % % % % % % % % % % % % % % % % % % % % % % % % 

\department{Katedra Softwarového Inženýrství}
\title{Systém správy úkolů pro jednotlivce a malé týmy}
\authorGN{Martin} %(křestní) jméno (jména) autora
\authorFN{Melka} %příjmení autora
\authorWithDegrees{Martin Melka} %jméno autora včetně současných akademických titulů
\supervisor{Ing. Josef Pavlíček, Ph.D.}
\acknowledgements{Doplňte, máte-li komu a za co děkovat. V~opačném případě úplně odstraňte tento příkaz.}

\abstractCS{Tato bakalářská práce se zabývá srovnáním existujících aplikací a tvorbou nové aplikace pro správu úkolů. Uživateli této aplikace budou jednotlivci a menší pracovní skupiny, a přidělovat a sledovat průběh práce na společných úkolech. Aplikace umožní lidem sdružovat se do skupin, spolupracovat na sdílených úkolech a zaznamenávat odvedenou práci. Součástí této práce je definice požadavků na aplikaci, srovnání navrhovaného řešení s existujícími aplikacemi, dále návrh, implementace, testování a nasazení aplikace. Výsledkem práce bude aplikační backend, vystavující funkcionalitu skrze REST rozhraní.}

\abstractEN{The aim of this thesis is to compare available software applications for task management and to subsequently create an original one. The users of this application will be individuals and small-scale workgroups, who need to assign responsibilities for and track the progress of shared tasks. The application will allow users to form groups, work together on shared tasks and report the work done on them. This thesis consists of a definition of application requirements, comparison of current task management solutions and design, implementation, testing and deployment of the proposed application. The result of this thesis will be an application backend, which exposes its functionality through a REST interface.}
\placeForDeclarationOfAuthenticity{V~Praze}
\declarationOfAuthenticityOption{4} %volba Prohlášení (číslo 1-6)
\keywordsCS{Správa úkolů, produktivita, organizace týmů, backend, REST, Java Spring Framework}
\keywordsEN{Task management, productivity, team organization, backend, REST, Java Spring Framework}

\begin{document}

% \newacronym{CVUT}{{\v C}VUT}{{\v C}esk{\' e} vysok{\' e} u{\v c}en{\' i} technick{\' e} v Praze}
% \newacronym{FIT}{FIT}{Fakulta informa{\v c}n{\' i}ch technologi{\' i}}

\begin{introduction}
	Výpočetní technika umožnila rozvoj rychlejší a efektivnější komunikace, práce a vůbec způsobu života. Je běžné mít svůj kalendář on-line a sdílet ho s ostatními, případně používat některý nástroj s funkcí úkolníčku. Zejména ve velkých firmách, kde existuje silná potřeba koordinovat úsilí mnoha lidí, tak vznikla poptávka po nástrojích pro správu úkolů, které by jim umožnily efektivnější rozdělování zodpovědností a práce. Řešení, které na tento popud vznikly, slouží právě potřebám velkých firem. Potřeby menších skupin a jednotlivců jsou ale jiné. Pro ně jsou tyto nástroje příliš komplexní, těžkopádné a nedostatečně intuitivní.

Tato práce se zabývá přehledem existujících řešení pro správu úkolů malých i velkých skupin a tvorbou backendu nové aplikace. Aplikace bude zaměřena na potřeby menších týmů a jednotlivců a bude umožňovat uživatelům vytvářet úkoly, sdílet je s ostatními uživateli, sdružovat se do skupin a zaznamenávat průběh práce na úkolech. 

V první části porovnám současná řešení pro správu úkolů z pohledu malých týmů a jednotlivců. 

Ve druhé části se zabývám analýzou problému. Na základě kladů a záporů jednotlivých řešení zmíněných v části první vytvořím požadavky na aplikaci ve formě uživatelských příběhů a tyto požadavky podrobím analýze.

Ve třetí části vypracuji návrh řešení aplikace. Zde vybírám technologii pro implementaci a popisuji architekturu aplikace, návrhový model tříd a databázový model.

Ve čtvrté části popisuji implementaci aplikace.

V páté části se věnuji testování aplikace za účelem zjištění její správné funkčnosti. Popíšu zde použité technologie a způsob testování.

V poslední, šesté části, vysvětlím, jak výslednou aplikaci nasadit a spustit.
\end{introduction}

\chapter{Současný stav}

Způsobů, jak řešit správu úkolů, existuje spoustu a liší se podle toho, kdo je má využívat. V této části práce představím několik zástupců pro každou ze tří kategorií. Těmi jsou:
\begin{enumerate}
  \item Řešení pro větší firmy s množstvím pracovníků
  \item Řešení pro střední a menší týmy, jednotky až desítky pracovníků
  \item Řešení pro jednotlivce
\end{enumerate}

Zástupce vybírám podle osobních zkušenosti a podle výsledků získaných z vyhledávače \texttt{Google}, na základě jejich pořadí a popularity mezi uživateli. U uvedených zástupců zvážím klady a zápory jak pro jejich cílovou skupinu, tak pro cílovou skupinu této práce, tj. menší týmy a jednotlivce.

Pracuji s tím, že pro cílovou skupinu této práce jsou důležitá následující kritéria:
\begin{enumerate}
	\item \textbf{Nevyžaduje vlastní infrastrukturu, rychlé zprovoznění} -- Uživatelé nechtějí spravovat vlastní hardware, kde by jim aplikace běžela. Začít používat aplikaci má být otázka maximálně minut.
	\item \textbf{Jednoduché používání} -- Relativně subjektivní kritérium; uživatel by se neměl ztratit ve funkcích aplikace a měl by být schopen rychle pochopit, jak s aplikací pracovat.
	\item \textbf{Absence nepotřebné funkcionality} -- 	Aplikace by měla obsahovat jen základní funkce, které uživatel využije. Větší množství funkcí, které uživatele nezajímají, ztěžují orientaci v aplikaci, což souvisí s předchozím bodem.
	\item \textbf{Použití zdarma} -- Aplikace by měla být použitelná zdarma. V případě, že se jedná o \textit{freemium} model, měla by její neplacená část stačit k běžnému používání a neomezovat výrazně uživatele.
	\item \textbf{Funkce} -- Aplikace má umožňovat spravovat úkoly jak pro jednotlivce, tak týmy. Mezi požadavky na funkce tedy patří možnost vytvářet osobní i sdílené úkoly, u nich uvádět datum \textit{deadlinu}, pracnost, prioritu; dále sdružovat se do skupin, v rámci skupin přiřazovat odpovědnosti za úkoly a zaznamenávat odpracovanou práci jednotlivých uživatelů. Aplikace by také měla uživateli doporučit, kterému úkolu se věnovat, v závislosti na jeho prioritě, pracnosti a času zbývajícímu do deadlinu.
	\item \textbf{REST API} -- Aplikace by měla nabízet rozhraní REST API pro možnost vlastní integrace na její funkce.
	\item \textbf{Open-source} -- Aplikace by měla mít veřejně dostupné zdrojové kódy.
\end{enumerate}
Uvedený přehled není vyčerpávající, věnuji se jen některým z těch nejznámějších řešení. 

\section{Řešení pro firmy}
\section{Řešení pro střední a menší týmy}
\section{Řešení pro jednotlivce}

\chapter{Analýza a návrh}

\chapter{Realizace}

\begin{conclusion}
	%sem napište závěr Vaší práce
\end{conclusion}

\bibliographystyle{csn690}
\bibliography{mybibliographyfile}

\appendix

\chapter{Seznam použitých zkratek}
% \printglossaries
\begin{description}
	\item[GUI] Graphical user interface
	\item[XML] Extensible markup language
\end{description}


% % % % % % % % % % % % % % % % % % % % % % % % % % % % 
% % Tuto kapitolu z výsledné práce ODSTRAŇTE.
% % % % % % % % % % % % % % % % % % % % % % % % % % % % 
% 
% \chapter{Návod k~použití této šablony}
% 
% Tento dokument slouží jako základ pro napsání závěrečné práce na Fakultě informačních technologií ČVUT v~Praze.
% 
% \section{Výběr základu}
% 
% Vyberte si šablonu podle druhu práce (bakalářská, diplomová), jazyka (čeština, angličtina) a kódování (ASCII, \mbox{UTF-8}, \mbox{ISO-8859-2} neboli latin2 a nebo \mbox{Windows-1250}). 
% 
% V~české variantě naleznete šablony v~souborech pojmenovaných ve formátu práce\_kódování.tex. Typ může být:
% \begin{description}
% 	\item[BP] bakalářská práce,
% 	\item[DP] diplomová (magisterská) práce.
% \end{description}
% Kódování, ve kterém chcete psát, může být:
% \begin{description}
% 	\item[UTF-8] kódování Unicode,
% 	\item[ISO-8859-2] latin2,
% 	\item[Windows-1250] znaková sada 1250 Windows.
% \end{description}
% V~případě nejistoty ohledně kódování doporučujeme následující postup:
% \begin{enumerate}
% 	\item Otevřete šablony pro kódování UTF-8 v~editoru prostého textu, který chcete pro psaní práce použít -- pokud můžete texty s~diakritikou normálně přečíst, použijte tuto šablonu.
% 	\item V~opačném případě postupujte dále podle toho, jaký operační systém používáte:
% 	\begin{itemize}
% 		\item v~případě Windows použijte šablonu pro kódování \mbox{Windows-1250},
% 		\item jinak zkuste použít šablonu pro kódování \mbox{ISO-8859-2}.
% 	\end{itemize}
% \end{enumerate}
% 
% 
% V~anglické variantě jsou šablony pojmenované podle typu práce, možnosti jsou:
% \begin{description}
% 	\item[bachelors] bakalářská práce,
% 	\item[masters] diplomová (magisterská) práce.
% \end{description}
% 
% \section{Použití šablony}
% 
% Šablona je určena pro zpracování systémem \LaTeXe{}. Text je možné psát v~textovém editoru jako prostý text, lze však také využít specializovaný editor pro \LaTeX{}, např. Kile.
% 
% Pro získání tisknutelného výstupu z~takto vytvořeného souboru použijte příkaz \verb|pdflatex|, kterému předáte cestu k~souboru jako parametr. Vhodný editor pro \LaTeX{} toto udělá za Vás. \verb|pdfcslatex| ani \verb|cslatex| \emph{nebudou} s~těmito šablonami fungovat.
% 
% Více informací o~použití systému \LaTeX{} najdete např. v~\cite{wikilatex}.
% 
% \subsection{Typografie}
% 
% Při psaní dodržujte typografické konvence zvoleného jazyka. České \uv{uvozovky} zapisujte použitím příkazu \verb|\uv|, kterému v~parametru předáte text, jenž má být v~uvozovkách. Anglické otevírací uvozovky se v~\LaTeX{}u zadávají jako dva zpětné apostrofy, uzavírací uvozovky jako dva apostrofy. Často chybně uváděný symbol "{} (palce) nemá s~uvozovkami nic společného.
% 
% Dále je třeba zabránit zalomení řádky mezi některými slovy, v~češtině např. za jednopísmennými předložkami a spojkami (vyjma \uv{a}). To docílíte vložením pružné nezalomitelné mezery -- znakem \texttt{\textasciitilde}. V~tomto případě to není třeba dělat ručně, lze použít program \verb|vlna|.
% 
% Více o~typografii viz \cite{kobltypo}.
% 
% \subsection{Obrázky}
% 
% Pro umožnění vkládání obrázků je vhodné použít balíček \verb|graphicx|, samotné vložení se provede příkazem \verb|\includegraphics|. Takto je možné vkládat obrázky ve formátu PDF, PNG a JPEG jestliže používáte pdf\LaTeX{} nebo ve formátu EPS jestliže používáte \LaTeX{}. Doporučujeme preferovat vektorové obrázky před rastrovými (vyjma fotografií).
% 
% \subsubsection{Získání vhodného formátu}
% 
% Pro získání vektorových formátů PDF nebo EPS z~jiných lze použít některý z~vektorových grafických editorů. Pro převod rastrového obrázku na vektorový lze použít rasterizaci, kterou mnohé editory zvládají (např. Inkscape). Pro konverze lze použít též nástroje pro dávkové zpracování běžně dodávané s~\LaTeX{}em, např. \verb|epstopdf|.
% 
% \subsubsection{Plovoucí prostředí}
% 
% Příkazem \verb|\includegraphics| lze obrázky vkládat přímo, doporučujeme však použít plovoucí prostředí, konkrétně \verb|figure|. Například obrázek \ref{fig:float} byl vložen tímto způsobem. Vůbec přitom nevadí, když je obrázek umístěn jinde, než bylo původně zamýšleno -- je tomu tak hlavně kvůli dodržení typografických konvencí. Namísto vynucování konkrétní pozice obrázku doporučujeme používat odkazování z~textu (dvojice příkazů \verb|\label| a \verb|\ref|).
% 
% \begin{figure}\centering
% 	\includegraphics[width=0.5\textwidth, angle=30]{cvut-logo-bw}
% 	\caption[Příklad obrázku]{Ukázkový obrázek v~plovoucím prostředí}\label{fig:float}
% \end{figure}
% 
% \subsubsection{Verze obrázků}
% 
% % Gnuplot BW i barevně
% Může se hodit mít více verzí stejného obrázku, např. pro barevný či černobílý tisk a nebo pro prezentaci. S~pomocí některých nástrojů na generování grafiky je to snadné.
% 
% Máte-li například graf vytvořený v programu Gnuplot, můžete jeho černobílou variantu (viz obr. \ref{fig:gnuplot-bw}) vytvořit parametrem \verb|monochrome dashed| příkazu \verb|set term|. Barevnou variantu (viz obr. \ref{fig:gnuplot-col}) vhodnou na prezentace lze vytvořit parametrem \verb|colour solid|.
% 
% \begin{figure}\centering
% 	\includegraphics{gnuplot-bw}
% 	\caption{Černobílá varianta obrázku generovaného programem Gnuplot}\label{fig:gnuplot-bw}
% \end{figure}
% 
% \begin{figure}\centering
% 	\includegraphics{gnuplot-col}
% 	\caption{Barevná varianta obrázku generovaného programem Gnuplot}\label{fig:gnuplot-col}
% \end{figure}
% 
% 
% \subsection{Tabulky}
% 
% Tabulky lze zadávat různě, např. v~prostředí \verb|tabular|, avšak pro jejich vkládání platí to samé, co pro obrázky -- použijte plovoucí prostředí, v~tomto případě \verb|table|. Například tabulka \ref{tab:matematika} byla vložena tímto způsobem.
% 
% \begin{table}\centering
% 	\caption[Příklad tabulky]{Zadávání matematiky}\label{tab:matematika}
% 	\begin{tabular}{|l|l|c|c|}\hline
% 		Typ		& Prostředí		& \LaTeX{}ovská zkratka	& \TeX{}ovská zkratka	\tabularnewline \hline \hline
% 		Text		& \verb|math|		& \verb|\(...\)|	& \verb|$...$|		\tabularnewline \hline
% 		Displayed	& \verb|displaymath|	& \verb|\[...\]|	& \verb|$$...$$|	\tabularnewline \hline
% 	\end{tabular}
% \end{table}
% 
% % % % % % % % % % % % % % % % % % % % % % % % % % % % 

\chapter{Obsah přiloženého CD}

%upravte podle skutecnosti

\begin{figure}
	\dirtree{%
		.1 readme.txt\DTcomment{stručný popis obsahu CD}.
		.1 exe\DTcomment{adresář se spustitelnou formou implementace}.
		.1 src.
		.2 impl\DTcomment{zdrojové kódy implementace}.
		.2 thesis\DTcomment{zdrojová forma práce ve formátu \LaTeX{}}.
		.1 text\DTcomment{text práce}.
		.2 thesis.pdf\DTcomment{text práce ve formátu PDF}.
		.2 thesis.ps\DTcomment{text práce ve formátu PS}.
	}
\end{figure}

\end{document}
